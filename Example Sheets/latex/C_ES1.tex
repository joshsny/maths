\documentclass[a4paper]{article}

\def\npart{III}

\def\ntitle{Combinatorics}
\def\nsheet{I}

\def\ndate{\today}

\input{header}

\let\SO\undefined
\usepackage{tkz-graph}

\newcommand{\shadow}{\partial}
\renewcommand{\P}{\mathbb P}

\begin{document}
	\begin{titlepage}
  \begin{center}
%    \includegraphics[width=0.6\textwidth]{logo.jpg}\par
    \vspace{2cm}
    {\scshape\huge University of \par
      \Huge Cambridge \par}
    \vspace{1cm}
    {\scshape\huge Mathematics Tripos \par}
    \vspace{2cm}
    {\huge Part \npart \par}
    \vspace{0.6cm}
    {\Huge \bfseries \ntitle \par}
    \vspace{0.6cm}
    {\huge Example Sheet \nsheet \par}
    \vspace{1.2cm}
    {\Large\ndate \par}
    \vspace{2cm}
    
    {\large \emph{Solutions by } \par}
    \vspace{0.2cm}
    {\Large \scshape Joshua Snyder}
 \end{center}
\end{titlepage}

	\subsection*{Introduction}
	These are written solutions to \ntitle \ Example Sheet \nsheet. Solutions are written based on those seen in examples classes and may contain errors, likely due to the author. Solutions may be incomplete and do not usually include starred questions. These are to be used as a reference for revision \textbf{after} examples classes and should never be used beforehand. Doing so will severely impair your ability to learn and study mathematics.
	\subsection*{Questions}

	% Question 1
	\begin{question}[Question 1]
	Let $P = (V, <)$ be a finite poset. Recall that a subset $U \subset V$ is a chain if any two elements of $U$ are comparable, and it is an antichain if no two elements of $U$ are comparable. Show that the maximal size of an antichain in $P$ is equal to the minimal number of chains in $P$ that cover $V$.
	\end{question}
	\begin{proof}
	Let $N_1 = $ maximum size of antichain, $N_2 = $minimum number of chains that cover $V$.
	\begin{description}
	\item \underline{$N_2 \geq N_1$} Given $A_1, A_2, ..., A_{N_2}$ minimal number of chains covering $V$. Any antichain $B$ can contain at most one element from each $A_i$ so $N_1 \geq |B| \geq N_2$.
	\item \underline{$N_1 \geq N_2$} We prove this by induction on $n$, the size of the partially ordered set. If $P$ is empty the theorem is vacuously true. Thus, assume $P$ has atleast one element and let $a$ be a maximal element in $P$ which exists since $P$ is finite. By induction, assume $\exists k \ : \ P^{'} := P \ {a}$ can be covered by $k$ disjoint chains $C_1,..,C_k$ and there is an antichain $A_0$ of size atleast $k$. Have $A_0 \cap C_i \not = \emptyset$. Let $x_i$ be the maximal element of $C_i$ belonging to an antichain of length atleast $k$.\\
	\begin{remark}[Claim] Let $A_0 = \{x_1,x_2,...,x_k\}$, then $A$ is an antichain
	\begin{proof}[Proof of Claim]\renewcommand{\qedsymbol}{}
	Let $A_i$ be an antichain of size $k$ that contains $x_i$, fix $i \not = j$
  arbitrarily. Then $A_i \cup C_j \not = \emptyset$. Suppose $y \in A_i \cup
  C_j$. Then $y \leq x_j$ since $x_j$ is maximal in $C_j$. Thus $x_i \not \geq
  x_j$ since $x_i \not \geq y$. Exchanging $i,j$ gives $x_j \not \geq x_i$. ...
	\end{proof}
	\end{remark}
	Now suppose $a \geq x_i$ for some $1 \leq i \leq k$. Then set
	\[K = \{a\} \cup \{z \in C_i : z \leq x_i\}\]
	Then by choice of $x_i$, $P \ K$ does not have an antichain of size $k$ and so by induction $P \setminus K$ can be covered by $k-1$ disjoint chains as $A \ {x_i}$ is an antichain of size $k-1$ in $P \setminus K$. Thus $P$ can be covered by $k$ disjoint chains.\\
	Else, suppose instead that $a \not \geq x_i$ for all $1 \leq i \leq k$. The $A \cup \{a\}$ is an antichain of size $k+1$ in $P$ and $P$ can be covered by $k+1$ chains $\{a\}, C_1, C_2,...,C_k$.
	\end{description}
	\end{proof}

	\begin{remark}
	This proof is tedious and a very difficult Question 1. The ideas are, however, important and should be understood.
	\end{remark}

	% Question 2
	\begin{question}[Question 2]
	Let $(V, <)$ be a finite ranked poset with non-empty level sets $V_0 V_1,....,V_n$. Suppose for $0 < i \leq n$ every $v \in V_i$ dominates exactly $d_i \geq 1$ elements of $V_{i-1}$, for $0 \leq i < n$ every $v \in V_i$ is dominated by exactly $e_i \geq 1$ elements of $V_{i+1}$, and the partial order on $V = \cup_0^n V_i$ is induced by these relations.\\
	Show that if $U \subset V$ is an antichain then
	\[\sum_0^n{\frac{|U \cap V_i|}{|V_i|}} \leq 1\]
	\end{question}
	\begin{idea}
	Count number of chains of maximal length in two ways
	\end{idea}
	\begin{proof}
	Must have $|V_i|e_i = |V_{i+1}|d_{i+1}$ for all $0 \leq i < n$. Thus there are $|V_0| e_0 e_1 ... e_{n-1} = d_1 ... d_k |V_k|e_k ... e_{n-1}$ chains of maximal length in $V$.\\
	For each maximal chain $C$ we have $|C \cap U| \leq 1$ as $U$ is an antichain. Every element in $V_k$ is contained in exactly $(d_k d_{k-1} ... d_1)(e_k ... e_{n-1})$ maximal chains.\\
	Putting both of these together gives:
	\[\sum_0^n | U \cap V_k | (d_k ... d_ 1)(e_k ... e_{n-1}) = \# \text{maximal chains} = |V_0|e_0 ... e_{n-1}\]
	which upon dividing the LHS by the RHS yields the required result.

	\end{proof}
	\begin{remark}
	Counting arguments like these are popular. The counting itself is not difficult, but knowing what to count often is.
\end{remark}

% Question 3
\begin{question}[Question 3]
  Let $\mathcal{F} \subset \P(n)$ be a Sperner family i.e. let $\mathcal{F}$ be
  such that $A \not \subset B$ whenever $A, B \in \mathcal{F}$, $A \neq B$. Show
  that
  \[\sum_{k=0}^n \frac{f_k}{{n \choose k}} \leq 1\]
  where $f_k$ is the number of $k$-sets in $\mathcal{F}$.
\end{question}

\begin{proof}[Solution 1]
  \begin{idea}
    Use the Local LYM inequality repeatedly.
  \end{idea}
  Let $\mathcal{A} \subset X^{(r)}$, by the Local LYM inequality we have that
  \[\frac{|\shadow{A}|}{{n \choose r-1}} \geq \frac{|A|}{{n \choose r}} \]
  i.e that the shadow of a set has higher density then the set itself.\\
  Let $\mathcal{F}_r = \mathcal{F} \cap X^{(r)}$ so that $|\mathcal{F}_r| = f_r$. Since $|\mathcal{F}_n| / {n \choose n} \leq 1$ we have that
  \[1 \geq \frac{|\shadow{\mathcal{F}_n} \cup \mathcal{F}_{n-1}|}{{n \choose
        n-1}} = \frac{|\shadow{\mathcal{F}_n}|}{{n \choose n-1}} +
    \frac{|\mathcal{F}_{n-1}|}{{n \choose n-1}} \geq \frac{|\shadow{\mathcal{F}_n}|}{{n \choose n}} +
    \frac{|\mathcal{F}_{n-1}|}{{n \choose n-1}} \]
  Where the second equality holds since the two sets are disjoint, else the
  family would not be Sperner. Repeating this gives the desired result.
\end{proof}

\begin{proof}[Solution 2]
  \begin{idea}
    Use the result from Question 2
  \end{idea}
  Let $V = \P(n)$ as in Question 2, ranked by inclusion. Then every set of size
  $k$ contains $k$ sets of size $k-1$ and is contained in $k+1$ sets of size
  $k+1$. Thus from Question 2, $|V_k| = {n \choose k}$ and $| U \cap V_k| = f_k$
  and so the result follows.
  \end{proof}
  \begin{proof}[Solution 3]
    \begin{idea}
      Pick a chain uniformly at random and use probability
    \end{idea}

    Pick a chain uniformly at random in $Q_n$. Take $A \in X^{(r)}$. Then the
    probability that $C$ coincides with $A$ is
    \begin{align}
      &\P(C \ \text{meets} \ A) = \frac{1}{{n \choose r}}\\
      &\implies \P(C \ \text{meets} \ \mathcal{F}_k) = \frac{f_k}{{n \choose k}}\\
      &\implies \P(C \ \text{meets} \ \mathcal{F}) = \sum_{k=o}^n \frac{f_k}{{n \choose k}}
    \end{align}
    from which the result follows since all probabilities are bounded above by
    $1$.
  \end{proof}
  \begin{remark}
    The third proof exhibits a useful idea. Picking at random and using
    probability to prove a result is a popular method in combinatorics and leads
    to the field known as Probabalistic Combinatorics.
  \end{remark}

  % Question 4
  \begin{question}[Question 4]
    Let $2 \leq 2r < n$ and let $\mathcal{F} = \mathcal{F}_r \cup
    \mathcal{F_{n-r}} \subset \P(n)$ be a Sperner family where $\mathcal{F}_r
    \subset X^{(r)}$, $\mathcal{F}_{n-r} \subset X^{(n-r)}$ and
    $|\mathcal{F}_r| = |\mathcal{F}_{n-r}| = m$. At most how large is $m$?
  \end{question}
  \begin{proof}
    \begin{idea}
      Use the fact that $\shadow^{n-2r}{\mathcal{F}_{n-r}}$ and $\mathcal{F}_r$
      are disjoint.
    \end{idea}
    We get that $|\shadow^{n-2r}{\mathcal{F}_{n-r}}| + |\mathcal{F}_r| \leq {n
      \choose r}$
    \[1 \geq \]
  \end{proof}
  \begin{remark}
    I cannot go from here to get more then $m \leq \frac{1}{2} \binom{n}{r}$
    which one can achieve directly from the LYM inequality. The reason for this
    is the only proof I can think of is essentially a proof of LYM for this
    particular case. See a bubble down Kruskal-Katona proof for help. The answer
    should be $\binom{n-1}{r-1}$
  \end{remark}

  % Question 5
  \begin{question}[Question 5]
    For $2 \leq r \leq \frac{n}{2}$, let $A \subset Xˆ{(r)}$ be an intersecting
    family. (Thus $A \cap B \neq \emptyset$, whenever $A, B \in \mathcal{A}$.)
    Deduce from the Kruskal-Katona Theorem that $\mathcal{A} \leq
    \binom{n-1}{r-1}$\\
    What is the maximal size of an intersecting family $\mathcal{A} \subset
    \P(X)$? What about in the case $A \subset Xˆ{(\leq r)}$
  \end{question}
  
  \begin{proof}
  For $A,B \in \mathcal{A}$, have $A \cap B \neq \emptyset$, i.e.\ $A \nsubseteq B^c$.
  Writing
  \begin{equation*}\overline{\mathcal{A}} \coloneqq \{A^c \mid A \in \mathcal{A}\} \subset X^{(n-r)},\end{equation*}
  this says that $\partial^{n-2r}\overline{\mathcal{A}}$ is disjoint from $\mathcal{A}$.
  Now suppose that $|\mathcal{A}| > \binom{n-1}{r-1}$.
  Then $|\overline{\mathcal{A}}| > \binom{n-1}{r-1} = \binom{n-1}{n-r}$, so by
  repeated application of the LYM inequality we have $|\partial^{n-2r} \overline{\mathcal{A}}| \geq \binom{n-1}{r}$.
  But
  \begin{equation*}
    \binom{n-1}{r-1} + \binom{n-1}{r} = \binom{n}{r}
  \end{equation*}
  i.e.\
  \begin{equation*}
    |\partial^{n-2r} \overline{\mathcal{A}}| + |\mathcal{A}| > |X^{(r)}|.
  \end{equation*}
  a contradiction.\\

  For $\mathcal{A} \subset \P(X)$ an upper bound is $2ˆ{n-1}$ since a set and
  it's complemement cannot both be in a Sperner family. To achieve this maximal
  bound, we can extend any Sperner family, but a trivial example is $\mathcal{A}
  = \{ A : 1 \in A\}$.\\

  For $A \subset Xˆ{(\leq r)}$, applying $|A \cap Xˆ{(s)}| \leq \binom{n-1}{s-1}$ for
  all $1 \leq s \leq r$ gives an upper bound of $\sum_{s=1}ˆ{r}
  \binom{n-1}{s-1}$. This can be achieved with $\mathcal{A} = \{ A : |A| \leq r
  \ \text{and} \ 1 \in A\}$ 
\end{proof}
\begin{remark}
  For the first part, the numbers \emph{had} to work as we get equality for $\mathcal{A} = \{A \in X^{(r)} \mid 1 \in A\}$.
\end{remark}
\end{document}
