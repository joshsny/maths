\documentclass[a4paper]{article}

\def\npart{III}

\def\ntitle{Mixing Times of Markov Chains}
\def\nsheet{I}

\def\ndate{\today}

\ifx \nauthor\undefined
  \def\nauthor{Joshua Snyder}
\else
\fi

\ifx \ntitle\undefined
  \def\ntitle{Template}
\else
\fi

\ifx \nauthoremail\undefined
  \def\nauthoremail{jjds2@cam.ac.uk}
\else
\fi

\ifx \ndate\undefined
  \def\ndate{\today}
\else
\fi

\title{\ntitle}
\author{\nauthor}
\date{\ndate}

%\usepackage{microtype}
\usepackage{mathtools}
\usepackage{amsthm}
\usepackage{stmaryrd}%symbols used so far: \mapsfrom
\usepackage{empheq}
\usepackage{amssymb}
\let\mathbbalt\mathbb
\let\pitchforkold\pitchfork
\usepackage{unicode-math}
\let\mathbb\mathbbalt%reset to original \mathbb
\let\pitchfork\pitchforkold
\usepackage{float}

\usepackage{imakeidx}
\makeindex[intoc]

%to address the problem that Latin modern doesn't have unicode support for setminus
%https://tex.stackexchange.com/a/55205/26707
\AtBeginDocument{\renewcommand*{\setminus}{\mathbin{\backslash}}}
\AtBeginDocument{\renewcommand*{\models}{\vDash}}%for \vDash is same size as \vdash but orginal \models is larger
\AtBeginDocument{\let\Re\relax}
\AtBeginDocument{\let\Im\relax}
\AtBeginDocument{\DeclareMathOperator{\Re}{Re}}
\AtBeginDocument{\DeclareMathOperator{\Im}{Im}}
\AtBeginDocument{\let\div\relax}
\AtBeginDocument{\DeclareMathOperator{\div}{div}}

\usepackage{tikz}
\usetikzlibrary{automata,positioning}
\usepackage{pgfplots}
%some preset styles
\pgfplotsset{compat=1.15}
\pgfplotsset{centre/.append style={axis x line=middle, axis y line=middle, xlabel={$x$}, ylabel={$y$}, axis equal}}
\usepackage{tikz-cd}
\usepackage{graphicx}
\usepackage{newunicodechar}

\usepackage{fancyhdr}

\fancypagestyle{mypagestyle}{
    \fancyhf{}
    \lhead{\emph{\nouppercase{\leftmark}}}
    \rhead{}
    \cfoot{\thepage}
}
\pagestyle{mypagestyle}

\usepackage{titlesec}
\newcommand{\sectionbreak}{\clearpage} % clear page after each section
\usepackage[perpage]{footmisc}
\usepackage{blindtext}

%\reallywidehat
%https://tex.stackexchange.com/a/101136/26707
\usepackage{scalerel,stackengine}
\stackMath
\newcommand\reallywidehat[1]{%
\savestack{\tmpbox}{\stretchto{%
  \scaleto{%
    \scalerel*[\widthof{\ensuremath{#1}}]{\kern-.6pt\bigwedge\kern-.6pt}%
    {\rule[-\textheight/2]{1ex}{\textheight}}%WIDTH-LIMITED BIG WEDGE
  }{\textheight}% 
}{0.5ex}}%
\stackon[1pt]{#1}{\tmpbox}%
}

%\usepackage{braket}
\usepackage{thmtools}%restate theorem
\usepackage{hyperref}

% https://en.wikibooks.org/wiki/LaTeX/Hyperlinks
\hypersetup{
    %bookmarks=true,
    unicode=true,
    pdftitle={\ntitle},
    pdfauthor={\nauthor},
    pdfsubject={Mathematics},
    pdfcreator={\nauthor},
    pdfproducer={\nauthor},
    pdfkeywords={math maths \ntitle},
    colorlinks=true,
    linkcolor={red!50!black},
    citecolor={blue!50!black},
    urlcolor={blue!80!black}
}

\usepackage{cleveref}



% TODO: mdframed often gives bad breaks that cause empty lines. Would like to switch to tcolorbox.
% The current workaround is to set innerbottommargin=0pt.

%\usepackage[theorems]{tcolorbox}





\usepackage[framemethod=tikz]{mdframed}
\mdfdefinestyle{leftbar}{
  %nobreak=true, %dirty hack
  linewidth=1.5pt,
  linecolor=gray,
  hidealllines=true,
  leftline=true,
  leftmargin=0pt,
  innerleftmargin=5pt,
  innerrightmargin=10pt,
  innertopmargin=-5pt,
  % innerbottommargin=5pt, % original
  innerbottommargin=0pt, % temporary hack 
}
%\newmdtheoremenv[style=leftbar]{theorem}{Theorem}[section]
%\newmdtheoremenv[style=leftbar]{proposition}[theorem]{proposition}
%\newmdtheoremenv[style=leftbar]{lemma}[theorem]{Lemma}
%\newmdtheoremenv[style=leftbar]{corollary}[theorem]{corollary}

\newtheorem{theorem}{Theorem}[section]
\newtheorem{proposition}[theorem]{Proposition}
\newtheorem{lemma}[theorem]{Lemma}
\newtheorem{corollary}[theorem]{Corollary}
\newtheorem{axiom}[theorem]{Axiom}
\newtheorem*{axiom*}{Axiom}

\surroundwithmdframed[style=leftbar]{theorem}
\surroundwithmdframed[style=leftbar]{proposition}
\surroundwithmdframed[style=leftbar]{lemma}
\surroundwithmdframed[style=leftbar]{corollary}
\surroundwithmdframed[style=leftbar]{axiom}
\surroundwithmdframed[style=leftbar]{axiom*}
\surroundwithmdframed[style=leftbar]{question}

\theoremstyle{definition}

\newtheorem*{definition}{Definition}
\surroundwithmdframed[style=leftbar]{definition}

\newtheorem*{slogan}{Slogan}
\newtheorem*{eg}{Example}
\newtheorem*{ex}{Exercise}
\newtheorem*{remark}{Remark}
\newtheorem*{notation}{Notation}
\newtheorem*{convention}{Convention}
\newtheorem*{assumption}{Assumption}
\newtheorem*{question}{Question}
\newtheorem*{answer}{Answer}
\newtheorem*{note}{Note}
\newtheorem*{idea}{Idea}
\newtheorem*{application}{Application}

\renewcommand*{\proofname}{Solution}

%operator macros

%basic
\DeclareMathOperator{\lcm}{lcm}

%matrix
\DeclareMathOperator{\tr}{tr}
\DeclareMathOperator{\Tr}{Tr}
\DeclareMathOperator{\adj}{adj}

%algebra
\DeclareMathOperator{\Hom}{Hom}
\DeclareMathOperator{\End}{End}
\DeclareMathOperator{\id}{id}
\DeclareMathOperator{\im}{im}
\DeclareMathOperator{\coker}{coker}
\DeclarePairedDelimiter{\generation}{\langle}{\rangle}

%groups
\DeclareMathOperator{\sym}{Sym}
\DeclareMathOperator{\sgn}{sgn}
\DeclareMathOperator{\inn}{Inn}
\DeclareMathOperator{\aut}{Aut}
\DeclareMathOperator{\GL}{GL}
\DeclareMathOperator{\SL}{SL}
\DeclareMathOperator{\PGL}{PGL}
\DeclareMathOperator{\PSL}{PSL}
\DeclareMathOperator{\SU}{SU}
\DeclareMathOperator{\UU}{U}
\DeclareMathOperator{\SO}{SO}
\DeclareMathOperator{\OO}{O}
\DeclareMathOperator{\PSU}{PSU}
\DeclareMathOperator{\Sp}{Sp}


%hyperbolic
\DeclareMathOperator{\sech}{sech}

%field, galois heory
\DeclareMathOperator{\ch}{ch}
\DeclareMathOperator{\gal}{Gal}
\DeclareMathOperator{\emb}{Emb}



%ceiling and floor
%https://tex.stackexchange.com/a/118217/26707
\DeclarePairedDelimiter\ceil{\lceil}{\rceil}
\DeclarePairedDelimiter\floor{\lfloor}{\rfloor}


\DeclarePairedDelimiter{\innerproduct}{\langle}{\rangle}

%\DeclarePairedDelimiterX{\norm}[1]{\lVert}{\rVert}{#1}
\DeclarePairedDelimiter{\norm}{\lVert}{\rVert}



%Dirac notation
%TODO: rewrite for variable number of arguments
\DeclarePairedDelimiterX{\braket}[2]{\langle}{\rangle}{#1 \delimsize\vert #2}
\DeclarePairedDelimiterX{\braketthree}[3]{\langle}{\rangle}{#1 \delimsize\vert #2 \delimsize\vert #3}

\DeclarePairedDelimiter{\bra}{\langle}{\rvert}
\DeclarePairedDelimiter{\ket}{\lvert}{\rangle}




%macros

%general

%divide, not divide
\newcommand*{\divides}{\mid}
\newcommand*{\ndivides}{\nmid}
%vector, i.e. mathbf
%https://tex.stackexchange.com/a/45746/26707
\newcommand*{\V}[1]{{\ensuremath{\symbf{#1}}}}
%closure
\newcommand*{\cl}[1]{\overline{#1}}
%conjugate
\newcommand*{\conj}[1]{\overline{#1}}
%set complement
\newcommand*{\stcomp}[1]{\overline{#1}}
\newcommand*{\compose}{\circ}
\newcommand*{\nto}{\nrightarrow}
\newcommand*{\p}{\partial}
%embed
\newcommand*{\embed}{\hookrightarrow}
%surjection
\newcommand*{\surj}{\twoheadrightarrow}
%power set
\newcommand*{\powerset}{\mathcal{P}}

%matrix
\newcommand*{\matrixring}{\mathcal{M}}

%groups
\newcommand*{\normal}{\trianglelefteq}
%rings
\newcommand*{\ideal}{\trianglelefteq}

%fields
\renewcommand*{\C}{{\mathbb{C}}}
\newcommand*{\R}{{\mathbb{R}}}
\newcommand*{\Q}{{\mathbb{Q}}}
\newcommand*{\Z}{{\mathbb{Z}}}
\newcommand*{\N}{{\mathbb{N}}}
\newcommand*{\F}{{\mathbb{F}}}
%not really but I think this belongs here
\newcommand*{\A}{{\mathbb{A}}}

%asymptotic
\newcommand*{\bigO}{O}
\newcommand*{\smallo}{o}

%probability
\newcommand*{\prob}{\mathbb{P}}
\newcommand*{\E}{\mathbb{E}}

%vector calculus
\newcommand*{\gradient}{\V \nabla}
\newcommand*{\divergence}{\gradient \cdot}
\newcommand*{\curl}{\gradient \cdot}

%logic
\newcommand*{\yields}{\vdash}
\newcommand*{\nyields}{\nvdash}

%differential geometry
\renewcommand*{\H}{\mathbb{H}}
\newcommand*{\transversal}{\pitchfork}
\renewcommand{\d}{\mathrm{d}} % exterior derivative

%number theory
\newcommand*{\legendre}[2]{\genfrac{(}{)}{}{}{#1}{#2}}%Legendre symbol

%algebraic geometry
\DeclareMathOperator{\Spec}{Spec}
\DeclareMathOperator{\Proj}{Proj}

\let\SO\undefined
\usepackage{tkz-graph}

\newcommand{\shadow}{\partial}
\renewcommand{\P}{\mathbb P}
\newcommand{\tv}{\text{TV}}
\newcommand{\tmix}{t_{\text{mix}}}

\begin{document}
\begin{titlepage}
  \begin{center}
%    \includegraphics[width=0.6\textwidth]{logo.jpg}\par
    \vspace{2cm}
    {\scshape\huge University of \par
      \Huge Cambridge \par}
    \vspace{1cm}
    {\scshape\huge Mathematics Tripos \par}
    \vspace{2cm}
    {\huge Part \npart \par}
    \vspace{0.6cm}
    {\Huge \bfseries \ntitle \par}
    \vspace{0.6cm}
    {\huge Example Sheet \nsheet \par}
    \vspace{1.2cm}
    {\Large\ndate \par}
    \vspace{2cm}
    
    {\large \emph{Solutions by } \par}
    \vspace{0.2cm}
    {\Large \scshape Joshua Snyder}
 \end{center}
\end{titlepage}

\section{Introduction}
These are written solutions to Mixing Times of Markov Chains Example Sheet 1. Solutions are based on those handed out by Samuel Thomas and are not endorsed by the lecturer nor necessarily correct.
\section{Questions}

% Question 1
\begin{question}[Question 1]
	Let \(P\) be the transition matrix of a Markov chain with values in \(E\) and let \(\mu\) and \(\nu\) be two
	probability distributions on \(E .\) Show that
  \[ \|\mu P-\nu P\|_{\mathrm{TV}} \leq\|\mu-\nu\|_{\mathrm{TV}} \text { . } \]
  Deduce that \(d(t)=\max _{x}\left\|P^{t}(x, \cdot)-\pi\right\|_{\mathrm{TV}}\) is decreasing as a function of \(t,\) where \(\pi\) is the invariant
  distribution.
\end{question}

\begin{proof}
  Since $P$ is a stochastic matrix, any eigenvalue $\lambda$ of $P$ satisfies
  $|\lambda| \leq 1$. To prove this suppose that $Pv = \lambda v$ and let
  $|v_k| = \max{|v_1|,|v_2|, \dots, |v_n|}$; so $|v_k| > 0$. Then
  \[|\lambda||v_k| = |\lambda v_k| = |(Pv)_k| = |\sum_1^n P_{k,j}v_j| \leq
    |v_k|\]
  since $P$ is stochastic. Therefore $|\lambda| \leq 1$ and so $\norm{P} \leq
  1$.\\
  Then we can apply the inequality $\norm{(\mu - \nu)P}_1 \leq \norm{P}_1
  \norm{\mu - \nu}_1$.\\
  For the last part, using the fact that $\pi P = P$ gives the result.
\end{proof}

% Question 2
\begin{question}[Question 2]
  Let $\Sigma = \prod_{i=1}^n \Sigma_i$ where $\Sigma_i$ are finite sets. For
  each $i$, let $\mu_i$ and $\nu_i$ be probability distributions on $\Sigma_i$
  and set $\mu = \prod_{i=1}^n \mu_i$ and $\nu = \prod_{i=1}^n \nu_i$. Show
  that
  \[\norm{\mu - \nu}_\tv} \leq \sum_{i=1}^n \norm{\mu_i - \nu_i}_
  \tv\]
\end{question}
\begin{proof}
  For each $i = 1, \dots, n$, let $(X_i, Y_i)$ be an optimal coupling of
  $(\mu_i, \nu_i)$. Let $X \defeq (X_1, \dots, X_n)$ and $Y \defeq (Y_1, \dots ,
  Y_n)$. Then $(X,Y)$ is a coupling of $(\mu, \nu)$. From this we get that
  \[\norm{\mu - \nu}_\tv \leq \P(X \neq Y) \leq \sum_1^n \P(X_i \neq Y_i) =
    \sum_1^n \norm{\mu_i - \nu_i}_\tv\]
\end{proof}

% Question 3
\begin{question}
  Let $X$ and $Y$ be Poisson random variables with parameters $\lambda$ and
  $\mu$ respectively. Writing $\mathcal{L}(X)$ and $\mathcal{L}(Y)$ for their
  laws, prove that
  \[\norm{\mathcal{L}(X) - \mathcal{L}(Y)}_\tv \leq |\lambda - \mu|\]
\end{question}
\begin{proof}[Solution 1]
  Take a coupling of $(X,Y)$ by taking first $Y \sim Poiss(\mu)$ and $Z \sim
  Poiss(\lambda - \mu)$ independently and setting $X = Y + Z$. Then we have
  \[\norm{\mathcal{L}(X) - \mathcal{L}(Y)}_\tv \leq \P(X \neq Y) = \P(Z \neq 0) =
    1 - e^{-(\lambda - \mu)} \leq |\lambda - \mu|\]
\end{proof}
\begin{proof}[Solution 2]
  WLOG $\lambda \geq \mu$.\\
  \begin{align*}
    \norm{\mathcal{L}(X) - \mathcal{L}(Y)}_\tv &= \frac{1}{2} \sum_{n \geq 1} \left| \frac{e^{-\lambda}\lambda^n}{n!} - \frac{e^{-\mu}\mu^n}{n!} \right|\\
                                               &\leq \frac{1}{2} \sum_{n \geq 1} \frac{e^{-\lambda}}{n!}(\lambda^n - \mu^n)\\
                                               &= \frac{1}{2} ( \lambda - \mu ) \sum_{n \geq 1}\frac{e^{-\lambda}}{n!}(\lambda^{n-1} + \lambda^{n-2}\mu + \dots + \mu^{n-1})\\
                                               &\leq \frac{1}{2} ( \lambda - \mu ) \sum_{n \geq 1}\frac{e^{-\lambda}}{n!}(n\lambda^{n-1})\\
                                               &\leq \frac{1}{2}(\lambda - \mu)\left(1 + \sum_{n \geq 1}\frac{e^{-\lambda} \lambda^n}{n!}\right) = \frac{1}{2} \cdot 2 \cdot (\lambda - \mu) = \lambda - \mu
  \end{align*}
\end{proof}  

% Question 4
\begin{question}[Question 4]
  Let $Y$ be a random variable with values in $\N$ which satisfies
  \[\P(Y = j) \leq c,\text{ for all } j > 0 \text{ and } \P(Y = j) \text{ is
      decreasing in } j,\]
  where $c$ is a positive constant. Let $Z$ be an independent random variable with
  independent values in $\N$. Prove that
  \[\norm{\P(X+Y = \cdot) - \P(Y = \cdot)}_\tv \leq c\E[Z]\]
\end{question}
\begin{proof}
  Since the law of $Y$ is decreasing, $\P(Y = j) \geq \P(Y + k = j)$ for all $k,
  j \geq 0$. Therefore
  \begin{align*}
    \norm{\P(Y+k = \cdot) - \P(Y = \cdot )}_\tv &= \sum_{j: \P(Y = j) \geq \P(Y+k
                                                  = j)}(\P(Y = j) - \P(Y+k = j))\\
                                                &= \sum_j (\P(Y = j) - \P(Y+k = j)\\
    &= 1 - \P(Y \geq k) = \P(Y \leq k) \leq ck
    \end{align*}
\end{proof}
\begin{remark}
  Using this definition of the total variation distance can be very helpful when
  we know that a probability distribution is monotone past a certain value,
  which a lot of distributions are.
\end{remark}

% Question 5
\begin{question}[Question 5]
  Let $X$ be a Markov Chain and let $W$ and $V$ be random variables taking
  values in $\N$ and suppose they are independent of $X$. Prove that
  \[\norm{\P(X_W = \cdot) - \P(X_V = \cdot)}_\tv \leq \norm{\P(W = \cdot) - \P(V
      = \cdot)}_\tv\]
\end{question}
\begin{proof}
  Let $(X_W, X_V)$ be the optimal coupling of $\mu = \P (W = \cdot)$ and $\nu =
  \P( V = \cdot )$. Then we have
  \[\norm{\P (X_W = \cdot ) - \P ( X_V = \cdot )}_\tv \leq \P (X_W \neq X_V)
    \leq \P (W \neq V)\]
\end{proof}

% Question 6
\begin{question}[Question 6]
  Let $G = (V, E)$ be a finite connected graph with maximal distance between any
  two vertices equal to $D$. Suppose that $X$ is a lazy simple random walk on
  $G$. Prove that for all $\epsilon < \frac{1}{2}$ we have
  \[\tmix(\epsilon) \geq \frac{D}{2}\]
\end{question}
\begin{idea}
  Consider $\bar{d}$ instead of $d$ for lower bounds.
\end{idea}
\begin{proof}
  If $t \leq \frac{D}{2}$ and $d(x,y) = D$ then $supp(P^t(x, \cdot)) \cap
  supp(P^t(y, \cdot )) = \emptyset $ i.e they never meet and therefore
  $\bar{d}(t) = 1$.\\
  Therefore $d(t) \geq \frac{1}{2}\bar{d}(t) = \frac{1}{2}$ and the result follows.
\end{proof}

% Question 7
\begin{question}[Question 7]
Let $X$ be a Markov chain in $E$ with transition matrix $P$ and invariant
distribution $\pi$. Let $A \subset E$ be a subset with $\pi(A) \geq
\frac{1}{8}$. Let $\tau_A = \inf{t \geq 0: X_t \in A}$. Prove that there exists
a positive constant $c$ such that
\[\tmix\left( \frac{1}{4} \right) \geq c \max_x \E_x[\tau_A]\]
\end{question}
\begin{idea}
  Haven't hit a set of size $\frac{1}{8}$, so you know you are lying in only
  $\frac{7}{8}$ and so you cannot be mixed. Take as the distinguishing set the
  bad $\frac{1}{8}$.
\end{idea}
\begin{proof}
  Let $t = \tmix\left( \frac{1}{16} \right)$ then $d(t) \leq \frac{1}{16}$ and
  hence for all sets $A$ we have $P^t(x, A) \geq \pi(A) - \frac{1}{16}$.\\
  Let $A$ be such that $\pi(A) = \frac{1}{8}$, then $P^t(X, A) \geq
  \frac{1}{16}$.\\
  Hence $\tau(A) \lessapprox t \cdot Geom \left( \frac{1}{16} \right)$ since
  $\tau_A \leq \inf{k \geq 0 : X_t \in A} \lessapprox Geom \left( \frac{1}{16}
  \right)$ from the above.\\
  Therefore $\E[\tau_A] \lessapprox t$. Using the submultiplicativity of
  $\bar{d}$ and the inequality $d(t) \leq 2\bar{d}(t)$, the claim is deduced.
  \end{proof}
% Question 13
\begin{question}[Question 13]
	Let $P$ be a transition matrix of a finite reversible chain with invariant distribution $\pi$. Using the Cauchy-Schwarz inequality or otherwise prove that for all $x,y$ and all $t$
	\[ \frac{P^{2t}(x, y)}{\pi (y)} \leq \sqrt{\frac{P^{2t}(x,x)}{\pi (x)} \cdot \frac{P^{2t}(y,y)}{\pi (y)}} \ \text{and} \ P^{2t+2}(x,x) \leq P^{2t} (x,x)\]
\end{question}
\begin{proof}[Solution 1]
	\begin{align*}
    \left( \frac{P^{2t}(x, y)}{\pi (y)} \right)^2 &= \sum_{z} \frac{P^t(x,z)P^t(z,y)}{\pi(y)}\\
                                                  &= \sum_{z}\frac{P^t(x,z)P^t(z,y)}{\pi(z)}\\
                                                  &\leq \left( \sum_{z} \frac{P^{2t}(x,z)}{\pi(z)} \right) \left( \sum_{z} \frac{P^{2t}(y,z)}{\pi(z)} \right)\\
                                                  &\leq \left( \frac{P^{2t}(x,x)}{\pi(x)} \right) \left( \frac{P^{2t}(y,y)}{\pi(y)} \right)
	\end{align*}
	which upon taking square roots of both sides completes the proof.\\
	For the second part since $|\lambda_j| \leq 1$ we have that
	\begin{align*}
    \frac{P^{2t}(x,x)}{\pi (x)} &= \sum_{1}^{|\Sigma|} f_{j}^2 (x) \lambda_j^{2t}\\
                                &\geq \sum_{1}^{|\Sigma|} f_{j}^2 (x) \lambda_j^{2t+2}\\
                                &= \frac{P^{2t+2}(x,x)}{\pi (x)}
	\end{align*}
\end{proof}
\end{document}